\documentclass[itdr]{subfiles}

\begin{document}

\cleartoleftpage

\chapter{Playing the Game}

\section{Rules}
\index{Rules}
\index{Referee}

\paragraph{Saves}
\index{Saves}
Roll d20 equal or under the appropriate Ability Score to succeed. 1 is always a success and 20 is \mbox{always} a failure.

\vfill

\paragraph{Advantage and Disadvantage}
\index{Advantage}
\index{Disadvantage}
Whenever someone has increased or decreased odds of succeeding on a Save, the Referee may give them Advantage or Disadvantage. Roll twice and take the better or worse of the two rolls respectively. Advantage and Disadvantage cancel each other out.

\vfill

\paragraph{Taking your Turn}
\index{Combat}
\index{Turns}
In a combat situation, the Referee decides which side acts first. When this is unclear, player characters must pass \saves{DEX} to be able to act before their opponents. After such initial turn, all player characters act together as usual.

On their turn, characters can generally move (or change items they are holding instead) and carry out one action. All characters declare their intentions and after that the dice are rolled.

\vfill

\paragraph{Attacking}
\index{Attacks}
\index{Damage}
\index{Damage!bonus}
\index{Bonus Damage|see {Damage, bonus}}
Roll your weapon's Damage die, or for both weapons if wielding two, along with any bonus Damage dice you have. The highest single roll is identified, and the attack causes this much damage.

Ranged weapons cannot be used while engaged in melee combat.

\vfill

\paragraph{Ganging Up}
\index{Ganging Up}
\index{Bypassing HP}
When multiple attackers target an individual, they roll together and keep the highest result, plus 1 point of Damage for each additional attacker, up to +5. Once the attack has been resolved, the target cannot be attacked again until their next turn.

When some of these attacks directly target Ability Scores, they are grouped together by Ability Score targeted and resolved by the same Ganging Up rule, separately from normal attacks.

\begin{dbox}
\paragraph{Ganging Up: Easy Mode (optional)}
For a more ``cinematic'' combat feel, you may forgo bonus damage from additional attackers.
\end{dbox}
\break


\paragraph{Impaired and Enhanced Attacks}
\index{Attacks!impaired}
\index{Impaired Attacks|see {Attacks, impaired}}
Attacks that are Impaired, such as firing through cover or a resistant target, roll d4 Damage regardless of weapon, no bonus Damage dice allowed.

\index{Attacks!enhanced}
\index{Enhanced Attacks|see {Attacks, enhanced}}
Attacks that are Enhanced by a risky stunt or a vulnerable target gain bonus d12 Damage die.

\vfill
\paragraph{Manoeuvres}
\index{Manoeuvres}
Instead of making a normal attack, you may spend your turn trying to carry out another manoeuvre, such as knocking an opponent down, snatching an item or fleeing. In these cases, the side most at risk makes a Save to avoid consequences.

\vfill
\paragraph{Armour}
\index{Armour}
Armour subtracts its score from result of any Damage rolls against the wearer.

If the Damage bypasses HP, it is still affected by target's Armour, unless stated otherwise.

Total Armour score for a creature cannot exceed 3.

\vfill
\paragraph{Mounted Combat}
\index{Combat!mounted}
\index{Mounted Combat|see {Combat, mounted}}

Mounted troops in melee gain +1 Armour and bonus weapon Damage die against unmounted opponents.

\vfill
\paragraph{Damage}
\index{Damage}
\index{Hit Points}
When taking damage, you lose that many Hit Points. If you have any HP left, then the attack was mostly avoided or only a minor wound was inflicted.

When you run out of HP, any remaining Damage is removed from your STR score. Now you must pass a \save{STR} to avoid Critical Damage.

\vfill
\paragraph{Blast Damage}
\index{Damage!blast}
\index{Blast Damage|see {Damage, blast}}
Blast attacks affect all targets in the appropriate area, rolling once for each target. If unsure how many targets are affected, roll Damage die.

\vfill
\paragraph{Critical Damage}
\index{Damage!critical}
\index{Critical Damage|see {Damage, critical}}
Characters that take Critical Damage are unable to take further action until they are tended to by an ally and have a Rest. If they are left untended to for an hour, they die.

\vfill
\paragraph{Ability Score Loss}
\index{Ability Score Loss}
\index{Healing}
The character dies at STR~0. At DEX~0 or WIL~0 the character is paralysed or catatonic respectively, cannot act until Healing and must be carried to safety.

\vfill
\paragraph{Death}
\index{Death}
When a character dies, the player creates a new character and the Referee finds a way to have them join the group as soon as possible. Alternatively, the player may take control of a Hireling or Apprentice.

\vfill
\paragraph{Morale}
\index{Morale}
The leader of a group must pass a \save{WIL} to avoid their group being routed when they lose half of their total numbers. Lone combatants must pass this Save when they are reduced to 0hp. This applies to opponents and allies but not player characters. Mindless or fearless opponents are exempt as well.

\vfill
\paragraph{Retreat}
\index{Retreat}
Fleeing to safety under pursuit requires a \save{DEX} and somewhere to run to.

\vfill
\paragraph{Rest}
\index{Rest}
A few minutes of rest and a swig of water recovers all of character's lost Hit Points. Resting may waste time or attract danger.

\vfill
\paragraph{Healing}
\index{Healing}
Ability Score Loss and other serious ailments require the aid of an Expert service or magic to recover.

\vfill
\paragraph{Assumed Ability Scores}
\index{Ability Scores!assumed}
Any Ability Score not listed is treated as 10.

\vfill
\paragraph{Reaction}
\index{Reaction}
When a reaction to a character is uncertain, pass a \save{WIL} to avoid a negative reaction.

\vfill
\paragraphsection{Conditions:}
\index{Conditions}

\index{Blinded|see {Conditions}}
\subparagraph{Blinded} creatures may require a \save{DEX} to carry out actions that rely on sight, their attacks are Impaired, and \saves{DEX} from external threats are rolled at Disadvantage.

\vfill
\index{Hidden|see {Conditions}}
\subparagraph{Hidden} creature's attacks are Enhanced, but any attack or other similar action unveils the attacker.

\vfill
\index{Invisible|see {Conditions}}
\subparagraph{Invisible} creature's attacks are Enhanced, attacks targeted at invisible creatures are Impaired.

\vfill
\index{Stunned|see {Conditions}}
\subparagraph{Stunned} creatures do not act on their turn, have disadvantage on \saves{DEX}, and attacks targeted at them are Enhanced.

\vfill
\index{Unconscious|see {Conditions}}
\subparagraph{Unconscious} creatures are reduced to 0hp.

\vfill
\break

\section{After the Adventure}
\index{Adventure}
\index{After the Adventure}
\index{Advancement}
\index{Leveling Up}

Generally, the goal of an Adventure is to find out about a mysterious environment, destroy a powerful threat, or seek out mysterious treasures.

\vfill
\subsection{Experience Levels}
\index{Experience Levels}
\index{Levels|see {Experience Levels}}
After completing the requirements for the next Experience Level, you may take a break from adventuring to reflect upon your experience. Describe what the character has been doing during this time, whether humble or grand. After that, you move to the next Experience Level. You cannot advance more than one Experience Level in a single session of play.

\index{Features}
When advancing to a new Experience Level, you choose a new Feature, gain d6hp and roll d20 for each Ability Score. If the roll is higher than the Ability Score, it increases by one to a maximum of 20.

\vfill
\paragraph{1. Novice}
\index{Novice|see {Experience Levels}}
You are ready for your first Adventure.

\vfill
\paragraph{2. Proven}
\index{Proven|see {Experience Levels}}
You have survived at least one Adventure to a dangerous place, returning to civilisation.

\vfill
\paragraph{3. Expert}
\index{Expert|see {Experience Levels}}
\index{Apprentice}
You have survived at least three Adventures since reaching Proven Level.

You can now take on an Apprentice created as a new character.

\vfill
\paragraph{4. Veteran}
\index{Veteran|see {Experience Levels}}
You have survived at least five Adventures since reaching Expert Level. You have an Apprentice that has reached Expert Level.

\vfill
\paragraph{5. Master}
\index{Master|see {Experience Levels}}
You have established or seized control of a Domain of at least one hundred people. You are granted a noble title or create your own. Other characters may assist you in achieving this goal, though you are the only one becoming a Master while doing so.

\vfill

\begin{dbox}
	\paragraph{Slower Experience Progression (optional)}
	If the adventure progression of 1--3--5 feels too quick for your group, use the progression of 3--5--7 instead.
\end{dbox}

\vfill

\end{document}
