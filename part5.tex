\documentclass[itdr]{subfiles}

\begin{document}

\chapter{Running the Game}

\paragraph{Knowing when to Roll}
\index{Rolls}
When a player describes what they want their character to do, you generally
have three options.
\begin{enumerate}
	\item It's something the character can do safely.
	\item It's not possible. Ask for another approach, perhaps giving suggestions.
	\item It might be possible but there's a risk. Roll dice.
\end{enumerate}

\paragraph{A Note on Risk}
\index{Risk}
Generally the Referee should make the players aware if they are taking a risk. A game should have surprises, but players should feel that their decisions in the game have led to the risk that lead to the nasty surprise.

For example, when the characters encounter a monster or hazard that is very likely to be able to kill them outright, the Referee should ensure that the players know this is a possibility. If they want to hack down a door with axes, they should know the noise is likely to alert anyone nearby. Assessing the risk against the possible reward is an important part of the game, so the players should always have what they need to make an informed choice.

\paragraph{Understanding Ability Scores}
\index{Ability Scores}
3: Human minimum, severely limited in this area.\\
10: An average human.\\
15: Excellent human ability.\\
20: The human peak, most exceptional geniuses etc.\\

\paragraph{Understanding Saves}
\index{Saves}
A save is made  when a character has put themselves at risk.

\subparagraph{STR Save:} Avoiding harm through exerting physical force or withstanding strain on your body.

\subparagraph{DEX Save:} Avoiding harm through quick reactions, whole-body control and grace.

\subparagraph{WIL Save:} Avoiding harm through focus and control over magic and yourself.

\paragraph{Understanding Damage Dice}
\subparagraph{Increasing/Decreasing specific Damage dice:} die size changes by one; e.g. instead of d6 Damage, you roll d8.

\subparagraph{Bonus weapon Damage dice:} roll these along with your weapon Damage die. If die size is not specified, it is equal to your weapon Damage die.

\paragraph{Luck Rolls}
\index{Rolls!luck}
\index{Luck Rolls|see {Rolls, luck}}
Sometimes you'll want an element of randomness without rolling a Save, particularly in situations dictated by luck or those that fall outside of the three Ability Scores. For these situations roll a d6. A low roll favours the players, and a high roll means bad luck for them. The Referee decides what a specific result means for the situation at hand.

\paragraph{Knowledge Rolls}
\index{Rolls!knowledge}
\index{Knowledge Rolls|see {Rolls, knowledge}}
Characters have a 2-in-6 chance to know something outside their area of knowledge and past experiences; experts have a 4-in-6 chance for their wide area of expertise, and know everything about their narrow specialization (e.g. History (Archaeology)).

\paragraph{How Much Damage?}
\index{Damage}
Damage from falling rocks, explosions and other sources outside of normal combat is typically between d4 and d12 and counts independently, unlike damage from usual attacks in combat.

Consider how it would affect an average person. A fall that is quite likely to injure an inexperienced character might cause d6 Damage but a huge rock that would crush most might do d12.

\index{Poison}
Poison might Impair attacks, cause Ability Score Loss, effects like Blindness, Disadvantage to certain Saves, etc., but usually only alive targets are affected.

\paragraph{The Core of Good Refereeing}
\index{Referee}
A good Referee gives the players interesting choices to make and ensures that these choices have a meaningful impact on the current situation and progress of the game.

\end{document}
