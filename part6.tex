\documentclass[itdr]{subfiles}

\begin{document}

\chapter{Treasure and Magic}
\index{Treasure}
\index{Money}
\index{Magic}

\paragraph{Riches}
Different types of treasure, from gems to artwork to functional items, have a certain value. Traders often want to haggle this price or, in the case of items worth thousands of Shillings, they may not be able to afford it at all.

\paragraph{Coins}
All coins are valued against the Silver-Standard Shilling (s). One Shilling gets you a decent bed, meal and drink for the night and is the amount a typical labourer earns in a week.

There are a huge variety of coins that are valued against the Shilling, with two being especially common.

\subparagraph{Copper Pennies} (p) are worth a tenth of a Shilling. One penny buys you a cheap drink in a bad tavern or passage on a ferry.

\subparagraph{Gold Guilders} (g) are worth one-hundred Shillings. One Guilder gets you a good horse, full set of armour or a valuable piece of jewellery.

\paragraph{Creating New Magic Spells}
\index{Spells}
Use Chapter 4 as a reference of power levels and possible effects when creating new spells.

\index{Spells!damage}
Rough damage estimate:
\begin{itemize}
	\item \textbf{Cantrips}: d4
	\item \textbf{\nth{1} Circle}: d4 to d6
	\item \textbf{\nth{2} Circle}: d6 to d8
	\item \textbf{\nth{3} Circle}: d8 to d10
	\item \textbf{\nth{4} Circle}: d10 to d12
	\item \textbf{\nth{5} Circle}: d12
\end{itemize}

Continuous and area-of-effect spells usually deal less damage then instant ones of the same Circle.

Appropriate saves against certain effects:
\begin{itemize}
	\item \textbf{STR:} physical obstacles, touch spells, metamorphosis and other bodily influences
	\item \textbf{DEX:} evasion, balance, extinguishing the flames
	\item \textbf{WIL:} illusions and mind control
\end{itemize}

\vfill
\break

\paragraph{Breaking the Rules}
\index{Rules}
Not all magic functions as that of Mystics. Magic can do anything and is not subject to limitations.

\paragraph{Magic Weapons and Armour}
\index{Weapons!magic}
\index{Magic Weapons|see {Weapons, magic}}
\index{Armour!magic}
\index{Magic Armour|see {Armour!magic}}
\index{Runic}
Weapons created with magical power often have Runic symbols engraved on them, telling their name, history and purpose. As well as having a Damage die increased by one (up to d10) and ignoring supernatural resistances, magical weapons will have an extra feature, such as bursting into flame when it draws blood or guiding the wielder towards gold. This will never be a matter of simply doing more damage, though some weapons may cause additional effects when they cause Critical Damage, such as turning the victim to stone.

Similarly, magic armour and shields will have an extra feature or offer greater protection against a specific source of damage.

\paragraph{Magic Items}
\index{Magic!items}
\index{Magic Items|see {Magic, items}}
Other magic items could include rings, cloaks, gloves, and pendants. These may grant a continual effect on the wearer or require activation. The effect will usually not be exactly the same as a spell but may be similar.

\subparagraph{Consumable Magic Items} such as potions will grant a one-off benefit to the consumer.

\index{Wands|see {Magic, items}}
\index{Rods|see {Magic, items}}
\subparagraph{Wands and Rods} have a limited and unknown number of charges. After the first use roll a d4 and write it down. Every time you use the item roll a d6. If you roll over this number, decrease it by one. On zero the item is drained and becomes useless.

\dimage{treasures}{112pt}

\end{document}
