\documentclass[itdr]{subfiles}

\begin{document}

\chapter{Hazards and Obstacles}
\index{Hazards}
\index{Obstacles}

\paragraph{Spotting Hazards}
As a general rule the presence of a trap or other hazard is always noticed by characters unless they are running, visually impaired or distracted. After this the characters may be harmed through further inaction or lack of caution. The players should consider creative ways of getting around a hazard or disarming it completely. Risky methods may call for a Save or Luck Roll.

\paragraph{Locked Doors}
\index{Doors}
Typically a locked door can be picked by someone with a lockpick, given some time. No Save is required unless there is a risk of triggering a trap, alerting foes or running out of time.

Attempts to use lockpicks and other equipment quickly under pressure generally require a \save{DEX} and may include having to light a torch while under attack or tying a rope before a friend plummets to their death.

Breaking down a door can similarly be completed without a Save, unless there are risks or pressure, which may require a \save{STR}. However, breaking down a door always causes lots of noise and can take a long time.

\paragraph{Random Encounters}
\index{Encounters}
\index{Random Encounters|see {Encounters}}
Anything mobile in an expedition site is unlikely to remain in one place all the time. As such, the Referee should consider having a chance of the party encountering someone or something. Making loud noises increases or decreases the chance of this happening, depending on the nature of the encounter.

When characters explore, rest, cast unprepared spells, or hesitate in a dangerous place, roll a d6.

\begin{dtable}[cL]
	\textbf{d6} & \textbf{Outcome} \\
	1	& Roll for a Random Encounter.\\
	2	& Roll for a Random Encounter. Give a sign that it is nearby or has passed through.\\
	3--6& Clear.\\
\end{dtable}

Delaying for long enough to have a meal or sleep provokes a d4 roll instead.

\section{Example Traps}
\index{Traps}
\index{Traps!example}

\subparagraph{Stupefying Dart Trap}~\\
A dart pipe is visible at the base of the chest. Triggered by opening the chest without taking appropriate precautions. Broken darts litter the floor of this room.
d8 Damage. d8 DEX loss on Critical Damage.

\subparagraph{Balancing Ledge}~\\
Must be crossed to reach whatever lies on the other side. Can be done quite safely without pressure, but when having to run or under attack make a \save{DEX} or fall to the lower level, requiring a rope to climb back up.

The lower level contains crocodiles (STR~13, DEX~5, WIL~5, 9hp, Armour~1, d8~Bite).

\subparagraph{Swinging Blade Trap}~\\
Eternally swinging over a corridor in a sequence. Can be blocked only by a very strong metal pole or other such objects.

\save{DEX} to pass through without harm, otherwise taking d10 Damage in passing through.

\subparagraph{Grasping Vines}~\\
Triggered on nearing strange-looking vines. Take d6 Damage each turn until you break free. \save{STR} to break free each turn, otherwise you are immobile. Highly flammable.

\subparagraph{Cage Pit}~\\
A trapdoor is visible unless the character is distracted, sprinting or the vision is impaired. Triggered by stepping onto the trapdoor.

Triggering the trap causes d8 Damage, a metal cage traps the victim until released with a key and an alarm alerts someone unpleasant.

\subparagraph{Traitor's Circle}~\\
Triggered by entering the circle marked with a symbol depicting a dagger being thrust into a heart.

\save{WIL} or immediately attack your closest ally, continuing until knocked unconscious. If you pass this Save, you are thrown out of the circle and take d6 Damage.

\end{document}
