\documentclass[itdr]{subfiles}

\begin{document}

\chapter{Example of Play}
%{\small
{\setstretch{0.92}

{\em Three player-characters and their hireling torch bearer have been delving deep into a strange underground complex they stumbled on in an inhospitable desert.}

\vspace{0.5em}

\subparagraph{Referee:} The base of the long staircase leads into a spectacular room, some 30~ft high and equally wide. Its walls look like an intricate mosaic but the tiles are constantly shifting in colour. Waves of differing hues wash across the walls and the centre of the floor is dominated by a six-foot-wide circular shaft.

\subparagraph{Ezekiel (Mystic):} {\em (Sketching down the room on his rough map)} Are there any exits other than the way we came?

\subparagraph{Referee:} Just the shaft in the middle of the room.

\subparagraph{Toku (Warrior):} Well, this is a dead end. My hunter's instincts were right!

\subparagraph{Ezekiel:} The walls look strange\ldots I'm being very careful not to touch them and tell my torch bearer to do the same.

\subparagraph{Toku:} Oh come on, we hired him because he's disposable! Maybe Uthred should try touching them.

\subparagraph{Uthred (Warrior):} I'm not scared of the wall, but I'm not stupid. I'll try tapping the wall with the handle of my axe.

\subparagraph{Referee:} The pattern of the tiles doesn't seem to respond, but as you're inspecting them more closely, you can feel that they're giving off slight heat.

\subparagraph{Uthred:} Enough to burn me?

\subparagraph{Referee:} Doesn't look like it, only slight heat.

\subparagraph{Uthred:} I place my hand boldly against the tiles.

\subparagraph{Referee:} As soon as Uthred's hand touches the wall, the shifting colours stop, and a pulsing blue pattern starts to radiate from around Uthred's hand.

\subparagraph{Ezekiel:} Stand by for his head exploding\ldots

\subparagraph{Uthred:} You worry too much! How do the tiles feel?

\subparagraph{Referee:} They feel much like you'd expect a smooth mosaic too, but they are giving off a faint warmth.

\subparagraph{Uthred:} Huh, weird. Well, I'll take my hand off the wall and go check out the shaft.

\subparagraph{Referee:} As soon as you remove your hand from the wall, it starts to shift colours again and you now see the tiled shape of a person, looking almost like your own reflection. Barely a second later, the room is filled with crackling noise and the tiled visage of Uthred somehow steps out of the wall, hefting the axe from its back and taking up a combat stance.

\subparagraph{Toku:} Right, I'm not giving this thing a chance to pull us into the wall or whatever it's going to try. I leap at it with my daggers.

\subparagraph{Referee:} What everyone else is doing?

\subparagraph{Uthred:} I'll have at it with my axe, trying to drive it away from Ezekiel and the torch-bearer.

\subparagraph{Ezekiel:} I'll enhance Toku's attack with my Guided Strike cantrip.

\subparagraph{Referee:} Okay, roll for damage.

\subparagraph{Toku:} {\em (Rolls 2d6 (two daggers) + d12 (enhanced attack), taking the highest result)} That's a 5!

\subparagraph{Uthred:} {\em (Rolls d8 (weapon damage) + d4 (bonus die), taking the highest result)} That's 6 damage!

\subparagraph{Referee:} {\em (Subtracts 7 (6 + 1 for the additional attacker) damage and notices that the opponent is now at 0hp, with 3 damage left over)} You kick the thing back, knocking it off balance and cutting through its side. {\em (Rolls a \save{STR} vs Critical Damage, succeeding)} The copy roars out in static fuzz but it's still standing.

\subparagraph{Uthred:} There's only room for one Uthred here!

\subparagraph{Referee:} The copy of Uthred drops its axe on the ground and reaches forward to try and grab Toku. Give me a \save{DEX}.

\subparagraph{Toku:} {\em (Rolls a \save{DEX})} Erm\ldots that's a 20.

\subparagraph{Referee:} {\em (Over the groans of the table)} The creature
grabs Toku and tries to push him against one of the walls. A blue pulsing pattern forms on its surface. A moment later the colours shift into Toku's shape and the copy steps forward from the wall. Over to you guys.

\subparagraph{Ezekiel:} I never thought I'd have to choose between killing Toku and Uthred. I'm going to use the Chill Touch spell I have held to destroy the copy of Uthred.

\subparagraph{Uthred:} And if he's still standing after that, I'll try to chop his head off!

\subparagraph{Referee:} It gets a \save{STR} to resist the effect {\em (Rolls a \save{STR})}, but it fails! Roll to see how much STR Uthred's copy loses.

\subparagraph{Ezekiel:} {\em (Rolls d4 for STR loss, as dictated by the
spell)} Four!

\subparagraph{Referee:} {\em (Checks his notes to see that this reduces the creature's STR to zero)} It's enough to drain the energy from this thing. The touch causes the colour to fade from the being as it falls motionless to the ground and snaps out of existence, completely destroyed.

\subparagraph{Uthred:} Yes!

\subparagraph{Referee:} Ezekiel, don't forget to take 2 damage from casting the spell. Also, you should be aware that you've been making quite a lot of noise in this room.

{\em (Secretly makes a Random Encounter Roll to see if any nearby monsters have noticed the noise. A roll of 1 indicates that encounter should happen, so he rolls on the hostile encounter table he has prepared for this area).}

\subparagraph{Ezekiel:} I don't like the sound of this.

\subparagraph{Referee:} You notice the sound of something descending the staircase. Remember that weird horse-like creature with skin like a super-hard tree bark you were ambushed by last session?

\subparagraph{Uthred:} Sure, we knocked it down that pit and fled like heroes.

\subparagraph{Referee:} Well, this thing looks almost identical, but rather than being horse-sized, it's large enough to be barely able to squeeze into the staircase. Its jaws look large enough to be able to swallow you whole and its forelegs end in grasping claws spanning some six feet. Needless to say, it's squeezing down the staircase with you in its sights and it doesn't seem friendly. {\em (Fails a \save{WIL} for the copy of Toku, as the sight of this thing is enough to scare it)} The copy of Toku sees this thing and immediately crawls back into the wall, fading into the tiles.

\subparagraph{Ezekiel:} I don't really like the idea of being swallowed whole. What are our chances of running through its legs?

\subparagraph{Referee:} It's pretty tightly packed into the stairway. If you want to try, it would certainly require a tough \save{DEX}.

\subparagraph{Uthred:} The smaller monster was afraid of fire, wasn't it? Perhaps we should send the torch bearer over to try and keep it at bay.

\subparagraph{Referee:} He looks pretty hesitant\ldots he'd need to pass a \save{WIL} to follow such a suicidal order. You never know, though, it could work!

\subparagraph{Toku:} Running past it and trying to scare it seem needlessly risky when we have a perfectly good exit right here!

\subparagraph{Uthred:} The shaft? Does it look like the creature could fit down there?

\subparagraph{Referee:} Unlikely, it's certainly too big to be able to do so easily.

\subparagraph{Ezekiel:} As suicidal as it sounds, it might be our best hope. Can I throw a coin or something into the shaft?

\subparagraph{Referee:} As you flick a half-shilling down the shaft, you hear a distant splash a few seconds later.

\subparagraph{Toku:} Water!

\subparagraph{Ezekiel:} That's optimistic\ldots how do we know it isn't
acid or something? I figure we can find a way to distract it long enough for us to escape back up the staircase.

\subparagraph{Referee:} While you're formulating this plan, the creature has managed to force itself into the room, brushing against the tiled wall, which sends out blue ripples.

\subparagraph{Uthred:} Oh crap, this isn't going to end well.

\subparagraph{Ezekiel:} Fine! Into the hole!

\subparagraph{Toku:} Trust me! I'll even leap it first.

\subparagraph{Referee:} You're all leaping down now?

{\em (The group all nod reluctantly)}

\subparagraph{Referee:} You plunge into the darkness of the shaft, falling for a few seconds before splashing into what feels like ice cold water, deep enough for you to fall into safely. The bearer's torch is extinguished and you're barely able to get your bearings in the pitch- black pit before you feel a tingling sensation over your bodies. \saves{WIL} all round!

{\em (Groans fill the table)}
}

\end{document}
