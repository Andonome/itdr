\documentclass[itdr]{subfiles}

\begin{document}
	
\cleartoleftpage

\chapterx{Appendix C: Class-ic Edition}
\index{Classes}
\index{Class-ic Edition|see {Classes}}

Instead of using Features and Backgrounds, you could use three original ``Into the Dungeon'' classes.

\begin{comment}
Every character chooses a Class from the following:

\textbf{Warrior}: An adept of martial training.

\textbf{Mystic}: A student of runic magic.

\textbf{Disciple}: A devoted follower of a set of teachings.
\end{comment}

\class{Warrior}
{\em Warriors are at their best in a combat situation. They hit the hardest, can take the most punishment, and control the battlefield with combat techniques.}

\paragraph{Defensive Training}
When rolling for the Hit Points (including your starting roll), roll twice and take the better result.

\paragraph{Offensive Training}
Gain bonus weapon Damage die.

\paragraph{Manoeuvres}
When performing an attack, you may add a Manoeuvre to it (push, trip, disarm, grapple for their next turn, etc.) The attack is carried out as usual, and the opponent must make a Save to avoid an additional effect described by you.

\paragraph{Advancement}
At Proven Level, gain a Follower with a Simple Weapon and Shield each time you visit a friendly settlement. At Expert, this is increased to d4, and at Veteran --- to d6. Your Followers all count as Novice Warriors, but cannot gain further Experience Levels. At Expert, you gain an Apprentice.

\vfill

\class{Mystic}
{\em Mystics study the science of magic. They decode the arcane Runes that give instructions of the precise methods of casting spells.}

\paragraph{Runic}
You can read and speak Runic. The language is particularly found on ancient scrolls, spell tomes, and magical artefacts.

\paragraph{Spellcasting}
You have a Mystic's Focus and Tome containing instructions for two Cantrips and six \nth{1} Circle Spells. Choose a Signature Spell (see Chapter~4: Magic).

\paragraph{Advancement}
Add a new Cantrip and three Spells (of a Circle equal or lower to your Experience Level) to your Tome. From Expert onwards, take on an Apprentice. Choose a new Signature Spell.

\break

\index{Disciple}
\class{Disciple}
{\em Disciples follow a particular Creed which guides their way of life. In return, they can perform daily Rituals and carry a Symbol that becomes imbued with power.}

\paragraph{Creed}
Choose a Creed to follow. This determines what Rituals you can perform, your Symbol, and the laws you must live by.

If you break any of your Creed laws, you must atone by actively enforcing each law of your Creed. Any benefits from Rituals or Symbols are immediately lost until the atonement is complete.

\paragraph{Symbol}
A Disciple's symbol bestows certain powers as long as they remain blessed.

\paragraph{Rituals}
Each Ritual you know can be activated once each day as an action.

\paragraph{Advancement}
Gain the advancement Creed benefit. At Expert Level, gain d4 Followers with Simple Weapons each time you visit a friendly settlement; you may now take on an Apprentice in your Creed.

\vfill

\subsection{Followers and Apprentices}
Followers have average Ability Scores and 3hp, do not have a Class nor advance in Experience Levels.

You can have up to WIL~/~2 (rounded down)\\Followers at a time. You do not need to take all of your Followers on every Adventure, but you are responsible for their food, shelter, equipment, etc.

You can only have one Apprentice at a time, created as a new character of your class.

\break

\section{Creeds}
\index{Creed}

\creed{The Ancient Word}
{\em\begin{itemize}
		\item Do not suffer disrespect of any Gods, alive or dead.
		\item Work to bring subjects of all Gods together.
\end{itemize}}

\subparagraph{Symbol --- Marble Staff}: You are able to command, but not create, lightning, water, and fire as you wish. If thrown at an enemy, these will cause d6 Blast Damage.

\subparagraph{Commanding Ritual}: You bellow a single word of power. The target must pass a \save{WIL} or obey: approach, halt, flee, etc.

\subparagraph{Wrathful Ritual}: Striking an opponent or structure with this staff while unleashing an ancient word of power causes d10 Damage and ignores Armour.

\subparagraph{Advancement}: You can target 1d6~(Proven), 2d6~(Veteran), or 3d6~(Master) creatures with your Commanding Ritual.

\vfill

\creed{The Closed Circle}
{\em\begin{itemize}
		\item Carry no possessions besides your robes.
		\item Partake of no luxury or desire.
\end{itemize}}

\subparagraph{Symbol --- Plain Robes}: These give you Armour~2 and your unarmed attacks strike for d8 Damage.

\subparagraph{Purity Ritual}: Ignore the next attack or Spell against you.

\subparagraph{Deadly Ritual}: Next time your target makes a Save against Critical Damage caused by you, it fails.

\subparagraph{Advancement}: Attack 2~(Proven), 3~(Veteran), or 4~(Master) targets each turn.

\vfill

\creed{The Dream Painters}
{\em\begin{itemize}
		\item Never refuse to paint out a story.
		\item Do not use your illusions to harm the innocent.
\end{itemize}}

\subparagraph{Symbol --- Brush Pendant}: You can conjure illusions with sound, smell, and heat, that last until touched.

\subparagraph{Artist's Ritual}: Your next illusion persists even when touched, but vanish when attacked.

\subparagraph{Veil Ritual}: Make target invisible until touched.

\subparagraph{Advancement}: Your illusions can cause Damage up to d6~(Proven), d8~(Veteran), or d10~(Master).

\vfill
\break

~\vspace{0.75em}

\creed{The Forgotten Watcher}
{\em\begin{itemize}
		\item Seek out all knowledge, nothing is forbidden.
		\item Show no mercy to your fellow man.
\end{itemize}}

\subparagraph{Symbol --- Mark of the Eye}: You can read any language, including Runic. You can cast spells from scrolls and tomes but cannot use a Mystic's Focus.

\subparagraph{Ritual of Secrets}: You immediately cast a spell that you have seen cast today without needing to read it.

\subparagraph{Calling Ritual}: You summon and control a floating glowing eye that you can see through as your own. You may exert yourself for d4 Damage (ignoring Armour, at 0hp Critical Damage is avoided by a \save{WIL}) to have the eye lash out with a bolt of fire for d8 Fire Damage.

\subparagraph{Advancement}: Choose a \nth{1}~(Proven), \nth{2}~(Veteran), or \nth{3}~(Master) Circle Spell to be able to cast as an action.

\vspace{1em}

\creed{The Grey Mourners}
{\em\begin{itemize}
		\item Honour the dead and guide their souls.
		\item Do not harm the restless or woken dead.
\end{itemize}}

\subparagraph{Symbol --- The Pale Book}: Undead creatures will not harm you and you may speak with them.

\subparagraph{Restful Ritual}: Repeating the final line of this Ritual over a body prevents it from being resurrected or turned into undead and allow a single question to the departing soul.

\subparagraph{Guiding Ritual}: An immaterial spirit aids you however you wish but cannot interact with the material world and can only communicate with you.

\subparagraph{Advancement}: Control 1~(Proven), 1d6~(Veteran), or 3d6~(Master) undead creatures that fail a \save{WIL}. They do not benefit from Rest or Healing.

\vfill
\break

\creed{The Iron Judges}
{\em\begin{itemize}
		\item Allow no injustice to occur.
		\item Do not kill.
\end{itemize}}

\subparagraph{Symbol --- Iron Rod}: \save{WIL} to avoid the effects of any Spell against you. Strikes for d6.

\subparagraph{Truth Ritual}: The next target you touch with your Symbol must answer the next question truthfully.

\subparagraph{Redemption Ritual}: Restore a being that died recently to life as long as they have not broken the Creed leading to their death.

\subparagraph{Advancement}: When you pass a \save{WIL} (from your Iron Rod) against a \nth{1}~(Proven), \nth{2}~(Veteran), or \nth{3}~(Master) Circle Spell, reflect it on the caster.

\vfill

\creed{The Old Faith}
{\em\begin{itemize}
		\item Obey and protect the natural order.
		\item Shun steel and other unnatural materials.
\end{itemize}}

\subparagraph{Symbol --- Wooden Cudgel}: You can talk with animals or plants and they will not harm you. Strikes for d6.

\subparagraph{Guardian Ritual}: A single animal serves you unquestioningly for the rest of the day.

\subparagraph{Vengeful Ritual}: Gain the senses of a beast.

\subparagraph{Advancement}: Animals or plants that fight alongside you gain bonus d6~(Proven), d8~(Veteran), or d10~(Master) Damage die.

\vfill

\creed{The Primal Zealots}
{\em\begin{itemize}
		\item Make your ancestors proud and never surrender.
		\item Use of spells and magical items is dishonourable.
\end{itemize}}

\subparagraph{Symbol --- Ancestral Totem}: Choose one of the following (or make your own) and gain its aspect:
\vspace{-0.3em}\begin{itemize}
	\item \textbf{Bear} --- +1 Armour;
	\item \textbf{Stag} --- run twice as fast;
	\item \textbf{Wolf} --- bonus weapon Damage die.
\end{itemize}
\vspace{-0.3em}
\subparagraph{Guiding Ritual}: You automatically succeed on your next Save.

\subparagraph{Raging Ritual}: Until the end of combat, you have advantage on \saves{STR} and can attack two targets per turn. This effect ends if you have not attacked or taken Damage since your last turn.

\subparagraph{Advancement}: You have a 1-in-6 (Proven), 2-in-6 (Veteran), or 3-in-6 (Master) chance to repeat your Raging Ritual when you take Damage in combat for the first time.

\vfill

\creed{The Shadow Stealer}
{\em\begin{itemize}
		\item Kill only your target.
		\item Never give up on a target.
\end{itemize}}

\subparagraph{Symbol --- Ivory Locket}: A beloved item, strand of hair, or object similarly linked to a person, may be placed within. The person is now considered your target. Your Damage against the target is always Enhanced.

\subparagraph{Stalker's Ritual}: You are able to glimpse at your target for a few seconds and become aware of the direction of their location and state of alertness.

\subparagraph{Mercy Ritual}: The next target you kill is only sent into a deep coma for the rest of the day.

\subparagraph{Advancement}: Gain a bonus d6~(Proven), d8~(Veteran), or d10~(Master) Damage die on your Enhanced attacks.

\vfill

\creed{The Silver Order}
{\em\begin{itemize}
		\item Obey the law wherever you are.
		\item Protect the good, Smite the wicked.
\end{itemize}}

\subparagraph{Symbol --- Silver Mace}: When you defeat an enemy, one ally recovers d6 HP.

\subparagraph{Shielding Ritual}: Add d6 to HP of your allies until the next Rest.

\subparagraph{Smiting Ritual}: You can turn any attack against an enemy into a Smite, gaining bonus d4 Damage die. If it kills the target, you can repeat this ritual.

\subparagraph{Advancement}: Your Smite die increases to d6~(Proven), d8~(Veteran), or d10~(Master).

\vfill

\creed{The Sun King}
{\em\begin{itemize}
		\item At least one act of charity each day.
		\item Do not give up on a good cause.
\end{itemize}}

\subparagraph{Symbol --- Golden Sun}: Repels unnatural creatures that fail a \save{WIL}.

\subparagraph{Sunlight Ritual}: Touching a target immediately restores one Ability Score fully.

\subparagraph{Burning Ritual}: Water is blessed, running clean and acting as Fire Oil against unnatural enemies.

\subparagraph{Advancement}: Blast unnatural enemies for a bonus d6~(Proven), d8~(Veteran), or d10~(Master) Damage die at range.

\vfill
\break

\creed{The Swordmasters}
{\em\begin{itemize}
		\item Only kill in a fair fight.
		\item Do not use ranged weapons or magic of any sort.
\end{itemize}}

\subparagraph{Symbol --- Master Sword and Armour}: This two-handed sword (d8) and ornate armour (2) are both required to benefit from Rituals.

\subparagraph{Duellist Ritual}: Until the end of an unaided combat with a single opponent, your melee attacks are Enhanced.

\subparagraph{War Ritual}: When you kill an opponent in melee next time, your allies' melee attacks are enhanced until your next turn.

\subparagraph{Advancement}: Gain bonus d6~(Proven), d8~(Veteran), or d10~(Master) Damage die on melee attacks.

\vspace{1em}

\creed{The Third Eye}
{\em\begin{itemize}
		\item Do not allow your Crystal to come to harm.
		\item Do not knowingly allow your mind to be tainted by magic or false gods.
\end{itemize}}

\subparagraph{Symbol --- Mind Crystal}: This shard of crystal floats at your will. You can move it and other objects remotely.

\subparagraph{Mind Stab Ritual}: Cause d8 Damage to one target. You may repeat this ritual today if you take d4 Damage (ignoring Armour) immediately.

\subparagraph{Autohypnosis Ritual}: The next time you take Critical Damage or Ability Score Loss, you may ignore it with a \save{WIL}.

\subparagraph{Advancement}: Project a message to someone (Proven), share senses with someone (Veteran), or read someone's surface thoughts (Master).

\vfill
\break

\creed{The Violet Masks}
{\em\begin{itemize}
		\item Seek out new experiences every day.
		\item Be humble and enforce humbleness on others.
\end{itemize}}

\subparagraph{Symbol --- Violet Mask}: You may sneak in plain sight as if you were in shadow. You may still require a Save, but you can attempt normally impossible manoeuvres.

\subparagraph{Disguise Ritual}: Others believe you are someone else if they fail a \save{WIL} at a Disadvantage.

\subparagraph{Muse's Ritual}: When you perform during a Rest, your allies are Healed.

\subparagraph{Advancement}: When you Save against Critical Damage cause your attacker d6~(Proven),~d8 (Veteran), or d10~(Master) Damage.

\vspace{1em}

\creed{The War Sages}
{\em\begin{itemize}
		\item Ask for no payment for service in battle.
		\item Teach those who wish to learn the art of your weapon.
\end{itemize}}

\subparagraph{Symbol --- Weapon}: The Disciple's weapon of choice counts as their Symbol. Choose to deal Damage as Fire, Cold, Lightning or any other Damage type with each attack.

\subparagraph{Ritual of Destruction}: Your weapon's next attack ignores Armour or any other type of resistance.

\subparagraph{Binding Ritual}: You read a Cantrip in Runic and bind it to your weapon. Whenever you attack, you may cast it on the target as well as attacking normally. The Cantrip remains bound to the weapon until a new one replaces it.

\subparagraph{Advancement}: You can use \nth{1}~(Proven), \nth{2}~(Veteran), or \nth{3}~(Master) Circle Spell in your Binding Ritual.

\vfill

\begin{comment}

\creed{???}
{\em\begin{itemize}
\item ???
\item ???
\end{itemize}}

\subparagraph{Symbol --- ???}:

\subparagraph{??? Ritual}:

\subparagraph{??? Ritual}:

\subparagraph{Advancement}:

\end{comment}
%\dimage{warsage}{306pt}
\end{document}
