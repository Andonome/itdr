\documentclass[itdr]{subfiles}

\begin{document}

\chapter{Monsters}
\index{Monsters}

Monsters are, by their very nature, different to people and animals. Thus they often have special abilities outside of their Ability Scores. An expedition site should contain mostly unique monsters but some examples are given in this section.

\paragraph{Hit Points}
Most creatures have between 1d6 and 5d6 HP. Remember that Hit Points are not purely the ability to absorb physical damage but also the monster's cunning and skill in avoiding harm.

\paragraph{Killing Monsters}
Monsters are treated exactly the same as characters other than noted exceptions.

\paragraph{Magic}
While some monsters may use Spells in the same way as Mystics, some are able to use spells without a Tome or Focus. Monsters do not need to follow the rules.

\paragraph{Armour}
Use character armour as a guide for how to represent monsters with tough hides or those large enough to be able to shrug off most weapons.

\paragraph{Damage}
Most monsters cause d6 Damage if nothing is mentioned. Some have a bigger Damage die or even bonus Damage dice.

\paragraph{Ability Score Loss and Death Attacks}
Particularly deadly creatures may reduce the target's Ability Score if they cannot make a Save, often resulting in a horrible fate if the score is reduced to zero.

\paragraph{A Note on Ability Scores}
When assigning Ability Scores 20 should generally be considered the maximum. A huge monster may look like it should have a STR of 30 or more, but consider that large creatures may not fight all that well. They should instead have their size represented by dealing more Damage and having higher Armour score.

\vfill
\break

\section{Monster Conversion}
\index{Monsters!conversion}

\subsection*{D\&D 5e}

\subparagraph{HP:} 1hp per HD. Maximum of 30.
\subparagraph{Armour:} Increase by 1 for noted armour, extreme resilience, and each size category above medium.
\subparagraph{Ability Scores:} Directly transferable, use CHA for WIL. Maximum of 20.
\subparagraph{Attacks:} Start at d6. Increase by one die for each size category above medium and once more if they wield a heavy weapon. No multi-attacks.
\subparagraph{Vulnerability / Resistance:} Replace with Enhance / Impair respectively.

\subparagraph{Other editions:} Same as 5e except:
\subsection*{D\&D 4e}
\subparagraph{HP:} 1hp per Level. $\times$3 for Solo creatures, +1hp for Small or bigger creatures.
\subparagraph{Ability Scores:} Same as 5e, except:
\begin{itemize}
	\item --4~STR for Humanoids and Monstrosities
	\item --2~STR for Undead
	\item --4~DEX for Large or bigger creatures
	\item --2~DEX for Medium or smaller Humanoids and Undead
	\item --2~CHA for Monstrosities
\end{itemize}

\subsection*{D\&D 3e and 3.5e, Pathfinder}
\subparagraph{HP:} 1hp per HD. +1hp for Small or Medium creatures and +2hp for Large or bigger creatures, except Oozes.
\subparagraph{Ability Scores:} If STR is not specified --- below 10.

\subsection*{OD\&D, Basic D\&D, AD\&D}
\subparagraph{HP:} 1hp per HD. +1hp for Small and Medium creatures and Large or bigger Oozes; +2hp for Large or bigger creatures.\\
If no HD specified, HD=HP/8 (round down).
\subparagraph{Morale:} keep using 2d6, or convert it to d20:
\begin{dtable}[lccccccccccc]
	\textbf{2d6} & 2 & 3 & 4 & 5 & 6 & 7 & 8 & 9 & 10 & 11 & 12 \\
	\textbf{d20} & 1 & 2 & 3 & 5 & 7 & 9 & 13& 16& 18 & 19 & 20 \\
\end{dtable}
\vfill
\break

\section{Example Monsters}
\index{Monsters!example}

The Referee should use these examples as the guidance for creating their monsters.

\statpar{Brain Lord}
STR~14, DEX~14, WIL~20, 18hp.

Its psychic ability allows it to levitate, project itself to other realities and telepathically issue any command. If the target refuses to obey the command, they lose d8~WIL unless they pass a \save{WIL}.

\subparagraph{Mind Blast:} attacks the target's mind with psychic energy for d8 Damage. Critical Damage from this attack affects WIL instead of STR and is avoided by a \save{WIL}.

\subparagraph{Critical Damage in melee:} target has its brain extracted and eaten. The Brain Lord absorbs its recent memories.



\statpar{Filth Eater}
STR~16, DEX~6, WIL~5, 16hp, Armour~1, d6~Bite.

Big, stupid beasts that eat nearly anything they find. Much prefer dead food to alive. Can bark out a very limited vocabulary of common tongue but have little comprehension.

\subparagraph{Critical Damage:} target contracts filth fever unless they pass a \save{STR}. If they fail, then for the next 24 hours they are violently ill and do not get the normal benefits from resting.

\dimage{filtheater}{156pt}

\break

\statpar{Gazer}
STR~16, DEX~16, WIL~17, 20hp, Armour~1.

Actively seeks to destroy any other lifeforms. Magic does not work within the Gazer's sight. May fire two of the following beams at different targets
each turn.

\subparagraph{Telekinesis Beam:} Up to an elephant-sized target is lifted, moved, or thrown. Living targets thrown this way take d6 Damage but thrown objects may cause up to d12, depending on the size.

\subparagraph{Terror Beam:} \save{WIL} or be terrified. If you do anything on your next turn other than freeze or flee, you lose d6~WIL.

\subparagraph{Disintegration Beam:} d10 Damage. Anyone taking Critical Damage is turned to dust. Will completely destroy static objects up to the size of an elephant.


\statpar{Gelatinous Cube}
STR~14, DEX~3, WIL~3, 16hp, Armour~2.

Appears as hazy, wet air until the observer is dangerously close to the cube. A chemical smell may betray its nature from further away. The cube is
attracted to noise and heat.

Does not perform normal attacks. Anyone that the cube moves over is engulfed unless they pass a \save{DEX} to jump aside, assuming there is room to. Those engulfed lose d8~DEX each turn and d6~STR every hour as they are digested. They cannot free themselves but must be pulled from the cube by other means. When the cube takes Critical Damage, it collapses into a puddle of sticky ooze.



\statpar{Gnoll}
STR~12, DEX~12, WIL~7, 9hp, Armour~2 (light armour and shield), axe (d6/d8), 3d6s in stolen coins and trinkets.

Their stink is highly recognisable and spreads throughout their surroundings.
1-in-6 chance of being non-hostile. Can be bargained with but are only really interested in meat, preferably alive.


\statpar{Goblin}
STR~8, DEX~14, WIL~8, 4hp, dagger (d6).

Mischievous creatures that can cast Cantrips. Can easily be bribed with items that they think are pretty.


\statpar{Hook Horror}
STR~15, DEX~8, WIL~6, 7hp, Armour~3, d10~Hooks.

The ten-foot tall horror stalks tunnels and caves, using its audible clicking as a form of echolocation. Its vision is very poor and it is easily disoriented by loud noises.

Anything dog-sized or smaller is potential food to be swallowed whole when dealt Critical Damage, causing d6~STR loss each round after swallowing. It will treat anything larger than this as a threat to its territory and fight fiercely, but will avoid anything larger than itself.


\statpar{Hootbear}
STR~15, DEX~6, WIL~5, 10hp, Armour~1, d10~Claws.

Constantly gives out subsonic hooting, used to sense its surroundings. Thus you can never sneak up on one unless its hearing is somehow impaired.

It can give out a single, boneshaking hoot that causes d6 Damage to everyone nearby. Anyone reduced to 0hp by this hoot is not at risk of a Critical Damage, but must pass a \save{STR} or be incapacitated.



\statpar{Landshark}
STR~17, DEX~8, WIL~8, 18hp, Armour~3, d8~Bite.

Carves through earth as if it were water, using this to lay in ambush for prey. If it fears for its life, a Landshark may cause a cave in. Falling rocks will cause d6 Damage but anyone that stays long enough to be buried takes d10 Damage. The Landshark will have burrowed away before this point.

\vfill
\break

\statpar{Orc}
STR~11, DEX~9, WIL~8, 5hp, Armour~2 (piecemeal armour and shield), martial weapon (d6/d8).

Amoral minions that are rarely seen outside of the service of some foul leader and vary hugely in appearance from one master to the next.


\statpar{Phase Panther}
STR~16, DEX~18, WIL~6, 13hp, d8~Clawed~Tentacles.

The beast's displaced image grants it Advantage on Critical Damage Saves. Will attack any other living things for fun without provocation.


\statpar{Purple Worm}
STR~20, DEX~3, WIL~5, 30hp, Armour~3, d10~Sting.

\subparagraph{Critical Damage:} target is stung, losing 3d6~STR.

Could try to swallow a medium or smaller creature. The target must pass a \save{DEX} or be swallowed whole, losing d10~DEX each turn and d8~STR every hour as they are digested. When rolling against Critical Damage, the worm must succeed on an additional \save{STR} or regurgitate all swallowed creatures.

\dimage{purpleworm}{234pt}

\break


\statpar{Red Dragon}
STR~20, DEX~10, WIL~12, 25hp, Armour~3, d10~Claws.

Can speak but generally chooses not to unless properly motivated.

\subparagraph{Fire Breath:} d6 Fire Damage to everyone within the blast. Also causes d6 Fire Damage at the end of their next turn until \save{DEX} is passed or any other way to put out the flames is found.

Red Dragons instinctively hoard treasure, particularly gold items. A dragon's hoard will be worth 5d20g. If properly harvested, dead dragon's parts will be worth 100g to specialist buyers.


\statpar{Rust Monster}
STR~10, DEX~12, WIL~5, 6hp, d6~Bite.

Does not normally attack. Able to turn metal into a rust-like dust, which it then consumes. If a melee opponent is carrying a metal weapon, shield, or armour, the Rust Monster will turn one of these to rust as an action unless they pass a \save{DEX}.

\statpar{Skeleton}
STR~10, DEX~13, WIL~12, Armour~2 (only against piercing attacks such as arrows and spears), 5hp, blunt sword (d6).

When a skeleton would be killed by physical attacks, it is smashed into at least two separate pieces. Unless they are kept apart, these will reform on the skeleton's next turn, remaining at 0hp. Each half will continue to fight, but the half without a sword causes only d4 Damage.


\vfill
\break

\dimage{snakedemon}{250pt}\vspace{-1em}

\statpar{Snake Demon}
STR~17, DEX~17, WIL~16, 18hp, Armour~1, six swords (6d6, can target multiple melee opponents).

Snake Demons are charged with overseeing hellish operations and leading lesser minions. They love single combat and will never turn down a duel. They can cast the following spells as an action.

\subparagraph{Soaring Flight:} The caster can fly quite swiftly until they touch the ground or take damage.

\subparagraph{Soul Barrier:} Ghostly visions of tortured souls form a barrier, screaming and lashing out. Anyone passing through this barrier takes d8 Damage and loses d6~WIL if they take Critical Damage.


\statpar{Stinkfrog}
STR~10, DEX~13, WIL~7, 6hp, Armour~1, spear (d8).

Attacks without provocation and generally try to lay an ambush for their targets. Amphibious and able to hop several times their own height. Natural animals show a strong animosity towards Stinkfrogs and will attack them in an attempt to drive them away.

\end{document}
