\documentclass[itdr]{subfiles}

\begin{document}

\cleartoleftpage

\chapter{Monsters}
\index{Monsters}

Monsters are, by their very nature, different to people and animals. Thus they often have special abilities outside of their Ability Scores. An expedition site should contain mostly unique monsters but some examples are given in this section.

\paragraph{Hit Points}
Most creatures have between 1d6 and 5d6 HP. Remember that Hit Points are not purely the ability to absorb physical damage but also the monster's cunning and skill in avoiding harm.

\paragraph{Killing Monsters}
Monsters are treated exactly the same as characters other than noted exceptions.

\paragraph{Magic}
While some monsters may use Spells in the same way as Mystics, some are able to use spells without a Tome or Focus. Monsters do not need to follow the rules.

\paragraph{Armour}
Use character armour as a guide for how to represent monsters with tough hides or those large enough to be able to shrug off most weapons.

\paragraph{Damage}
Most monsters cause d6 Damage if nothing is mentioned. Some have a bigger Damage die or even bonus Damage dice.

\paragraph{Ability Score Loss and Death Attacks}
Particularly deadly creatures may reduce the target's Ability Score if they cannot make a Save, often resulting in a horrible fate if the score is reduced to zero.

\paragraph{A Note on Ability Scores}
When assigning Ability Scores 20 should generally be considered the maximum. A huge monster may look like it should have a STR of 30 or more, but consider that large creatures may not fight all that well. They should instead have their size represented by dealing more Damage and having higher Armour score.


\vfill
\break

\section{Monster Conversion}

\subsection*{D\&D 5e}

\subparagraph{HP:} 1hp per HD. Maximum of 30.
\subparagraph{Armour:} Increase by 1 for noted armour, extreme resilience, and each size category above medium.
\subparagraph{Ability Scores:} Directly transferable, use CHA for WIL. Maximum of 20.
\subparagraph{Attacks:} Start at d6. Increase by one die for each size category above medium and once more if they wield a heavy weapon. No multi-attacks.
\subparagraph{Vulnerability / Resistance:} Replace with Enhance / Impair respectively.

\subparagraph{Other editions:} Same as 5e except:
\subsection*{D\&D 4e}
\subparagraph{HP:} 1hp per Level. $\times$3 for Solo creatures, +1hp for Small or bigger creatures.
\subparagraph{Ability Scores:} Same as 5e, except:
\begin{itemize}
	\item --4~STR for Humanoids and Monstrosities
	\item --2~STR for Undead
	\item --4~DEX for Large or bigger creatures
	\item --2~DEX for Medium or smaller Humanoids and Undead
	\item --2~CHA for Monstrosities
\end{itemize}

\subsection*{D\&D 3e and 3.5e, Pathfinder}
\subparagraph{HP:} 1hp per HD. +1hp for Small or Medium creatures and +2hp for Large or bigger creatures, except Oozes.
\subparagraph{Ability Scores:} If STR is not specified --- below 10.

\subsection*{OD\&D, Basic D\&D, AD\&D}
\subparagraph{HP:} 1hp per HD. +1hp for Small and Medium creatures and Large or bigger Oozes; +2hp for Large or bigger creatures.\\
If no HD specified, HD=HP/8 (round down).
\subparagraph{Morale:} keep using 2d6, or convert it to d20:
\begin{dtable}[lccccccccccc]
	\textbf{2d6} & 2 & 3 & 4 & 5 & 6 & 7 & 8 & 9 & 10 & 11 & 12 \\
	\textbf{d20} & 1 & 2 & 3 & 5 & 7 & 9 & 13& 16& 18 & 19 & 20 \\
\end{dtable}


\vfill
\break

\section{Ideas for Monster Creation}

\paragraph{Appearance and Behaviour}
Change visual appearance and behaviour of the existing monster. Changing size, or combining a couple of monsters into one is also a possibility.

\paragraph{Characters' Features}
Apply Features from Chapter 1 to non-player-characters and monsters, especially ``bosses''.

\paragraph{Effect on Critical Damage}
On a failed Critical Damage Save monster's target suffer some additional detrimental effect: illness, poison, ability score loss, or even death. Decide if target could Save against this.

\paragraph{Effect on Maximum Damage}
When monster inflicts maximum damage, the attack will bear some additional effect: grapple, poison, stun, etc. Decide if target could Save against this.

\paragraph{Pairing}
One type of monster enhances other type's attacks, provides protection, or some other advantage.

\paragraph{Power-ups}
Monster receives a power-up, a new attack, or changes tactics when it runs out of HP, saves against Critical Damage for the first time, takes Damage from a specific source, etc.

\paragraph{Special Abilities and Attacks}
Instead of its default attack monster can use a special one, be it a spell-like ability, or some other unusual effect. Some of these abilities might be ``passive'' (always enabled).

\paragraph{Tactics and Weapons}
Monsters might use unexpected combat tactic, especially when they fight in groups. If monster uses a weapon, change it to something unusual, or switch melee/ranged type.


\begin{dbox}
	~\\
	See \textbf{Appendix B: Bestiary} for example monsters and additional inspiration.
\end{dbox}

\break

\section{Example Monster Abilities}

\paragraph{Absorption}
When monster takes damage from a certain source (usually, elemental one), it restores monster's HP (or even STR) for the value of this damage instead.

\paragraph{Charge}
Monster rapidly closes distance to melee range. The target must succeed on a \save{DEX} or suffer increased damage and/or other effects.

\paragraph{Extra Limbs}
Monster have multiple Damage dice (still taking the highest one for a single target). Some monsters can even attack multiple opponents, dividing Damage dice between these attacks.

\paragraph{Grapple}
If target fails a \save{DEX} it can't move until a successful \save{STR or DEX} on following turns. Monsters can't attack with limbs they currently use for grapple, but strong ones might Damage the grappled target instead.

\paragraph{Indomitable}
Once per Rest, when taking Critical Damage, monster continues to fight as if it succeeded on this Save.

\paragraph{Swallow}
The target must succeed on a \save{DEX} or be swallowed whole, suffering Ability Score Loss (usually STR, DEX, or both) each following round. If monster suffers Critical Damage, it must pass an additional \save{STR} or regurgitate all swallowed creatures.

\paragraph{Volatile}
When monster takes Critical Damage it explodes, dealing Blast damage to everyone nearby.

\paragraph{Weakness}
When monster takes damage from a source of its weakness (even if this damage isn't highest one), monster loses some of its powers, becomes stunned, etc. Usually, such effect last for its next turn.

\vfill
\break

\section{Random Inspirations}

\begin{dtable}[cXcX]
	\textbf{d12} & \textbf{Nature} & \textbf{d12} & \textbf{Nature} \\ 
	1	&	artificial	&	7	&	magical	\\
	2	&	colonial	&	8	&	mutated	\\
	3	&	divine	&	9	&	natural	\\
	4	&	eldritch	&	10	&	primitive	\\
	5	&	ethereal	&	11	&	relict	\\
	6	&	fiendish	&	12	&	undead	\\
\end{dtable}

\begin{dtable}[cXcX]
	\textbf{d20} & \textbf{Appearance} & \textbf{d20} & \textbf{Appearance} \\ 
	1	&	bald	&	11	&	multicoloured	\\
	2	&	barbed	&	12	&	muscular	\\
	3	&	bloated	&	13	&	rotting	\\
	4	&	camouflaged	&	14	&	rusty	\\
	5	&	diseased	&	15	&	shadowy	\\
	6	&	furry	&	16	&	shimmering	\\
	7	&	gaunt	&	17	&	slimy	\\
	8	&	graceful	&	18	&	spotted	\\
	9	&	invisible	&	19	&	stinking	\\
	10	&	luminous	&	20	&	striped	\\
\end{dtable}

\begin{dtable}[cXcX]
	\textbf{d20} & \textbf{Trait} & \textbf{d20} & \textbf{Trait} \\ 
	1	&	acidic	&	11	&	multiplying	\\
	2	&	acoustic	&	12	&	parasite	\\
	3	&	adhesive	&	13	&	poisonous	\\
	4	&	armed	&	14	&	psychic	\\
	5	&	armoured	&	15	&	shelled	\\
	6	&	electric	&	16	&	shooting	\\
	7	&	fire	&	17	&	spewing	\\
	8	&	giant	&	18	&	swallowing	\\
	9	&	hypnotic	&	19	&	tiny	\\
	10	&	ice	&	20	&	vampiric	\\
\end{dtable}

\begin{dtable}[cXcX]
	\textbf{d20} & \textbf{Behaviour} & \textbf{d20} & \textbf{Behaviour} \\ 
	1	&	ambushing	&	11	&	musical	\\
	2	&	cunning	&	12	&	nocturnal	\\
	3	&	devouring	&	13	&	peaceful	\\
	4	&	elusive	&	14	&	raging	\\
	5	&	friendly	&	15	&	scavenging	\\
	6	&	gibbering	&	16	&	screaming	\\
	7	&	grappling	&	17	&	silent	\\
	8	&	greedy	&	18	&	skittish	\\
	9	&	insane	&	19	&	swarming	\\
	10	&	intelligent	&	20	&	whispering	\\
\end{dtable}

\begin{dtable}[cXcX]
	\textbf{d20} & \textbf{Locomotion} & \textbf{d20} & \textbf{Locomotion} \\ 
	1	&	aquatic	&	11	&	jumping	\\
	2	&	burrowing	&	12	&	rolling	\\
	3	&	climbing	&	13	&	running	\\
	4	&	crawling	&	14	&	shambling	\\
	5	&	fast	&	15	&	slithering	\\
	6	&	floating	&	16	&	slow	\\
	7	&	flowing	&	17	&	soaring	\\
	8	&	flying	&	18	&	subterranean	\\
	9	&	gliding	&	19	&	teleporting	\\
	10	&	immobile	&	20	&	walking	\\
\end{dtable}

\begin{dtable}[cXcX]
	\textbf{d20} & \textbf{Body} & \textbf{d20} & \textbf{Body} \\ 
	1	&	armless	&	11	&	one-armed	\\
	2	&	bodiless	&	12	&	one-legged	\\
	3	&	four-armed	&	13	&	stone	\\
	4	&	four-legged	&	14	&	tailed	\\
	5	&	legless	&	15	&	tentacled	\\
	6	&	limbless	&	16	&	two-armed	\\
	7	&	metal	&	17	&	two-headed	\\
	8	&	multi-armed	&	18	&	two-legged	\\
	9	&	multi-legged	&	19	&	winged	\\
	10	&	multi-limbed	&	20	&	wooden	\\
\end{dtable}

\begin{dtable}[cXcX]
	\textbf{d12} & \textbf{Head} & \textbf{d12} & \textbf{Head} \\ 
	1	&	blind	&	7	&	multi-eyed	\\
	2	&	brainless	&	8	&	multi-headed	\\
	3	&	deaf	&	9	&	mute	\\
	4	&	eyeless	&	10	&	one-eyed	\\
	5	&	headless	&	11	&	trunked	\\
	6	&	horned	&	12	&	two-headed	\\
\end{dtable}

\begin{dtable}[cXcl]
	\textbf{d20} & \textbf{Form} & \textbf{d20} & \textbf{Form} \\ 
1	&	amorphous	&	11	&	hedgehog, mole, shrew	\\
2	&	amphibian	&	12	&	humanoid	\\
3	&	animated	&	13	&	insect, arachnid	\\
4	&	bat	&	14	&	mollusc, worm	\\
5	&	bird	&	15	&	plant	\\
6	&	cat-like	&	16	&	reptile, serpent	\\
7	&	crustacean, myriapod	&	17	&	rodent, rabbit	\\
8	&	dog-like	&	18	&	ungulate (hoofed)	\\
9	&	fish	&	19	&	chimeric*	\\
10	&	fungi	&	20	&	shape-shifting*	\\
\end{dtable}
\em{* Roll two more times.}

\end{document}
